\documentclass[a4paper,UTF8]{article}
\usepackage{ctex}
\usepackage[margin=1.25in]{geometry}
\usepackage{color}
\usepackage{graphicx}
\usepackage{amssymb}
\usepackage{amsmath}
\usepackage{amsthm}
%\usepackage[thmmarks, amsmath, thref]{ntheorem}
\theoremstyle{definition}
\newtheorem*{solution}{Solution}
\newtheorem*{prove}{Proof}
\usepackage{multirow}
\usepackage{url}
\usepackage[colorlinks,urlcolor=blue]{hyperref}
\usepackage{enumerate}
\setcounter{secnumdepth}{4}
\renewcommand\refname{参考文献}
%--

%--
\begin{document}
\title{\textbf{《 计算机图形学》5月说明书}}
\author{学号,姓名,\href{mailto:xxx@xxx.com}{171240511@smail.nju.edu.cn}}
\maketitle

\section{开发环境}
\begin{itemize}
\item 系统: windows 10 1809
\item Anaconda3(5.3.1): 
	\begin{itemize}
		\item python 3.7.0
		\item numpy 1.15.1
		\item pillow 5.2.0
		\item pyqt 5.9.2
	\end{itemize}
\item IDE: Spyder 3.3.1
\item 解释器: ipython 6.5.0
\end{itemize}
\section{编译运行}
CLI编译运行的命令和讲义一样: python cg\_cli.py input\_path output\_path\\
\indent GUI的运行命令: python cg\_gui.py
\section{系统功能说明}
\subsection{命令行功能}
接受两个参数, 第一个参数为指令文件的绝对路径或相对路径, 第二个参数为画布保存位置的绝对路径.
实现讲义要求所有指令, 内容格式和讲义一致.
\subsection{图形功能}
\subsubsection{界面总览}

\end{document}